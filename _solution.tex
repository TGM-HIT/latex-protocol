%!TEX root=../protocol.tex	% Optional

\section{Konfiguration}
\subsection{Optionen}
\begin{tabularx}{\textwidth}{l X}
{\small \verb|landscape|}    & Richte das Dokument vertikal aus.\\
{\small \verb|minted|}       & Nutze das \texttt{minted} Paket zur Quelltextdarstellung.\\
{\small \verb|natbib|}       & Nutze NatBib zur Literaturverwaltung.\\
{\small \verb|nobib|}        & Deaktiviere die Literaturverwaltung.\\
{\small \verb|nofonts|}      & Nutze die Standard \LaTeX ~Schriftarten.\\
{\small \verb|noglo|}        & Deaktiviere Akronyme und das Glossar.\\
{\small \verb|nologos|}      & Zeichne keine Logos auf der Titelseite.\\
{\small \verb|notitle|}      & Zeichne keine Titelseite.\\
{\small \verb|notoc|}        & Zeichne kein Inhaltsverzeichnis.\\
{\small \verb|notable|}      & Zeichne keine Tabelle auf der Titelseite.
\end{tabularx}

\subsection{Variablen}
Variablen werden über Kommandos gesetzt, die als Parameter den gewünschten Wert erhalten.
\begin{code}{tex}
\myvariable{value}
\end{code}

\begin{tabular}{l l}
\textbf{Kommando} & \textbf{Inhalt}\\

{\small \verb|mysubtitle|} 	& Untertitel oder Zugehörigkeit\\
{\small \verb|mysubject|} 	& Thema / Fach, welches bearbeitet wird\\
{\small \verb|mycourse|} 	& Kurs / Klasse welche(r) besucht wird\\
{\small \verb|myteacher|} 	& Betreuende Lehrkraft\\
{\small \verb|myversion|} 	& Aktuelle Version des Dokuments\\
{\small \verb|mybegin|} 	& Datum des Beginns\\
{\small \verb|myfinish|} 	& Datum an dem die Arbeit beendet wurde
\end{tabular}

\newpage
\section{Kommandos}\label{sec:Kommandos}
\subsection{\texttt makefig}

\begin{listing}
\begin{code}{tex}
\makefig{images/logo-right.png}{height=2cm}{
    Mit Beschreibung und Label  % (Optional)
}{
    fig:caption-label           % (Optional)
}
\end{code}
\caption{\texttt makefig}
\label{lst:makefig}
\end{listing}
\makefig{images/logo-right.png}{height=2cm}{Mit Beschreibung und Label}{fig:caption-label}

\subsection{\texttt vardef}
\begin{listing}
\begin{code}{tex}
$$e^{i*\pi} = -1$$
\end{code}

$$e^{i*\pi} = -1$$

\begin{code}[firstnumber=last]{tex}
\begin{vardef}
    \addvardef{$e$}{Eulersche Zahl}
    \addvardef{$\pi$}{Kreiszahl}
    \addvardef{$i$}{Imagin\"are Einheit}
\end{vardef}
\end{code}

\begin{vardef}
    \addvardef{$e$}{Eulersche Zahl}
    \addvardef{$\pi$}{Kreiszahl}
    \addvardef{$i$}{Imaginäre Einheit}
\end{vardef}

\caption{\texttt vardef}
\label{lst:vardef}
\end{listing}

\newpage
\section{Anwendung}\label{sec:Anwendung}
Hier sollen die Schritte der Laborübung erläutert werden. Hier sind alle Fragestellungen der Lehrkraft zu beantworten. Etwaige Probleme bzw. Schwierigkeiten sollten ebenfalls hier angeführt werden.

In diesem Fall werden einige \LaTeX-Elemente dokumentiert, welche bei der Kreation von Protokollen behilflich sein könnten.

\subsection{Tabellen}
\begin{table}[H]
	\center
	\begin{tabular}{| c | l |}
		\hline Header 	& Kopf\\ \hline\hline
		\textbf{Lorem} 	& Ipsum dolor sit amet, consetetur sadipscing elitr\\ \hline
		\textbf{Ipsum} 	& At vero eos et accusam et justo duo dolores et ea rebum.\\
						& Stet clita kasd gubergren, no sea takimata sanctus\\ \hline
		\textbf{Dolor} 	& Consetetur sadipscing elitr, sed diam nonumy\\\hline
	\end{tabular}
	\caption{Tabular}
	\label{tab:tabular}
\end{table}

\subsubsection{TabularX}
TabularX erlaubt die Angabe der Größe der Tabelle und bietet zudem den Reihentyp \texttt{X}, der die verbleibende Größe neben anderen Reihen mit anderen \texttt{X} Reihen teilt.
\\\\
ACHTUNG: Die Verwendung von \verb|\codein|, \verb|\mintinline| oder \verb|\lstinline| ist in einer TabularX Umgebung nicht möglich!
\begin{table}
    \center
    \begin{tabularx}{\textwidth}{| c | X |}
        \hline Header 	& Kopf\\ \hline\hline
        \textbf{Lorem} 	& Ipsum dolor sit amet, consetetur sadipscing elitr\\ \hline
        \textbf{Ipsum} 	& At vero eos et accusam et justo duo dolores et ea rebum.\\
            			& Stet clita kasd gubergren, no sea takimata sanctus\\ \hline
        \textbf{Dolor} 	& Consetetur sadipscing elitr, sed diam nonumy\\\hline
    \end{tabularx}
    \caption{TabularX}
    \label{tab:tabularx}
\end{table}

\newpage
\subsection{Aufzählung}
\begin{itemize}
	\item Element einer Aufzählung
	\begin{itemize}
        \item Erstes eingerücktes Element einer Aufzählung
        \item Zweites eingerücktes Element einer Aufzählung
    \end{itemize}
\end{itemize}

\subsubsection{Outlines}
\begin{outline}
    \1 Element einer Aufzählung
        \2 Erstes eingerücktes Element einer Aufzählung
        \2 Zweites eingerücktes Element einer Aufzählung
\end{outline}

\subsection{Glossar}
Zur Verwaltung des Glossars wird standardmäßig die Datei \texttt{glossaries.tex} verwendet, wobei sowohl Definitionen als auch Akronyme definiert werden können.
\\\\
Als Beispiel wurde ein Akronym für \gls{ac-syt} und eine Definition zu \gls{ac-syt} selbst hinzugefügt.

\inputcode{tex}{glossaries.tex}

Im Dokument selbst kann ein Akronym mittels \verb|\gls{ac-syt}| verwendet werden. Beachte, dass ein Akronym welches bereits im Dokument verwendet wurde, bei der ersten Verwendung ausgeschrieben und danach immer gekürzt wird.
\\\\
Mit \verb|\gls{syt}| kann zum Beispiel eine Referenz zur Definition von \gls{syt} hinzugefügt werden.

\subsection{Zitate}
Zitate sollten gesammelt in der Datei \texttt{bib.bib} verwaltet werden.

\newpage
\subsection{Quelltext}
\begin{listing}
\ifminted   \codeline{tex}{\begin{code}[]{java}}
\else       \codeline{tex}{\\begin\{code\}[]\{java\}}\fi
\begin{code}[firstnumber=last]{java}
// Ich bin ein Kommantar!
public static void main(String[] args) {
    System.out.println("Ich bin ein Array!")
}
\end{code}
\ifminted   \codeline[firstnumber=last]{tex}{\end{code}}
\else       \codeline[firstnumber=last]{tex}{\\end\{code\}}\fi

\caption{Java Code}
\label{lst:java-code}
\end{listing}
~\\
Die Darstellung von Quelltext im Text ist über das Kommando \verb|\codein[options]{lang}{code}| möglich.
\\\\
Eine einzelne Zeile kann mittels
\ifminted   \codeline{tex}{\codeline[options]{lang}{code}}
\else       \codeline{tex}{\\codeline[options]\{lang\}\{code\}}\fi
eingefügt werden.

\subsubsection{Listings}
\begin{listing}
\ifminted   \codeline{tex}{\begin{lstlisting}[language=Java, caption=Java Lstlisting]}
\else       \codeline{tex}{\\begin\{lstlisting\}[language=Java, caption=Java Lstlisting]}\fi
\begin{code}[firstnumber=last]{java}
// Ich bin ein Kommantar!
public static void main(String[] args) {
    System.out.println("Ich bin ein Array!")
}
\end{code}
\ifminted   \codeline[firstnumber=last]{tex}{\end{lstlisting}}
\else       \codeline[firstnumber=last]{tex}{\\end\{lstlisting\}}\fi
\caption{Java Lstlisting}
\label{lst:java-lstlisting}
\end{listing}

\newpage
\subsubsection{Minted}
Benötigt die Option \texttt{minted}.
\paragraph{Umgebung}~\\
\begin{listing}
\ifminted   \codeline{tex}{\begin{minted}[options]{java}}
\else       \codeline{tex}{\\begin\{minted\}[]\{java\}}\fi
\begin{code}[firstnumber=last]{java}
// Ich bin ein Kommantar!
public static void main(String[] args) {
    System.out.println("Ich bin ein Array!")
}
\end{code}
\ifminted   \codeline[firstnumber=last]{tex}{\end{minted}}
\else       \codeline[firstnumber=last]{tex}{\\end\{minted\}}\fi
\caption{Minted Umgebung}
\label{lst:minted-env}
\end{listing}

\paragraph{Zeile}~\\
\begin{listing}
\begin{code}{tex}
\mint[options]{lang}|code|
\end{code}
\caption{Minted Einzeiler}
\label{lst:minted-line}
\end{listing}

\begin{listing}
\begin{code}{tex}
\mintinline[options]{lang}{code}
\end{code}
\caption{Minted Inline}
\label{lst:minted-inline}
\end{listing}